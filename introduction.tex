%%%%%%%%%%%%%%%%%%%%%%%%%%%%%%%%%%%%%%%%%%%%%%%%%%%%%%%%%%%%%%%%%%%%%%%%%%%%%%%%
\intro
%%%%%%%%%%%%%%%%%%%%%%%%%%%%%%%%%%%%%%%%%%%%%%%%%%%%%%%%%%%%%%%%%%%%%%%%%%%%%%%%

Основной темой данной работы является реализация системы классификации типов транспортных средств. На современном этапе развития технологий технического зрения и навигации существенным вопросом является возможность создания устойчивой системы определения класса транспортного средства. При правильной реализации подобной системы, можно добиться быстрого и эффективного средства анализа движения транспорта на дорогах. Такой результат связан с высокой скоростью обработки графических данных на современных вычислительных устройствах. 

При разработке подобных систем классификации возникают определенные технические трудности. Основные проблемы, которые приходится решать, включают в себя вопросы о компромиссе между быстродействием системы и её точностью, особенно в условиях работы на встраиваемых устройствах. Также, часто возникает проблема с обучением такой системы, если речь идет об алгоритмах машинного обучения.

Задача данной работы заключается в реализации системы классификации основных типов транспортных средств в условиях ограниченных вычислительных ресурсов. В работе также приведены основные данные об использовании различных методов классификации транспортных средств на основе систем технического зрения. Дана сравнительная оценка эффективности каждого из представленных методов. 

Некоторые из представленных методов технического зрения основываются на машинном обучении. С точки зрения машинного обучения задача классификации транспортного средства заключается в создании обучаемой системы на базе примеров, способной с достаточной точностью и скоростью определять признаки конкретного типа транспортного средства исходя из его изображения. Для этого используются нелинейные обучаемые системы, часто – искусственные нейронные сети. 

В основном, методы классификации транспортных средств можно разделить на: аппаратные и программные методы классификации. Аппаратные методы основаны на использовании различных инструментов, таких как магниты, радары и инфракрасные детекторы. У аппаратных методов есть определенные недостатки:
%
\begin{itemize*}
  \item Они могут иметь большие размеры и их трудно обслуживать
  \item Высока стоимость установки
  \item Высокая стоимость оборудования
  \item Ограниченная информация
\end{itemize*}
%

Программные методы классификации в основном полагаются на анализ видеопотока. В то время как такие методы лишены большинства недостатков представленных выше, они все еще имеют ограничение по быстродействию и точности обработки. В данной работе будут затронуты исключительно программные методы классификации.

