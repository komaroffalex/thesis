%%%%%%%%%%%%%%%%%%%%%%%%%%%%%%%%%%%%%%%%%%%%%%%%%%%%%%%%%%%%%%%%%%%%%%%%%%%%%%%%
\intro
%%%%%%%%%%%%%%%%%%%%%%%%%%%%%%%%%%%%%%%%%%%%%%%%%%%%%%%%%%%%%%%%%%%%%%%%%%%%%%%%

Основной темой данной работы является реализация системы классификации типов транспортных средств. На современном этапе развития технологий технического зрения и навигации существенным вопросом является возможность создания устойчивой системы определения класса транспортного средства. При правильной реализации подобной системы, можно добиться быстрого и эффективного средства анализа движения транспорта на дорогах. Такой результат связан с высокой скоростью обработки графических данных на современных вычислительных устройствах. 

При разработке подобных систем классификации возникают определенные технические трудности. Основные проблемы, которые приходится решать, включают в себя вопросы о компромиссе между быстродействием системы и её точностью, особенно в условиях работы на встраиваемых устройствах. Также, часто возникает проблема с обучением такой системы, если речь идет об алгоритмах машинного обучения.

Задача данной работы заключается в реализации системы классификации основных типов транспортных средств в условиях ограниченных вычислительных ресурсов. В работе также приведены основные данные об использовании различных методов классификации транспортных средств на основе систем технического зрения. Дана сравнительная оценка эффективности каждого из представленных методов. 

Некоторые из представленных методов технического зрения основываются на машинном обучении. С точки зрения машинного обучения задача классификации транспортного средства заключается в создании обучаемой системы на базе примеров, способной с достаточной точностью и скоростью определять признаки конкретного типа транспортного средства исходя из его изображения. Для этого используются нелинейные обучаемые системы, часто – искусственные нейронные сети. 

В основном, методы классификации транспортных средств можно разделить на: аппаратные и программные методы классификации. Аппаратные методы основаны на использовании различных инструментов, таких как магниты, радары и инфракрасные детекторы. У аппаратных методов есть определенные недостатки:
%
\begin{itemize*}
  \item Они могут иметь большие размеры и их трудно обслуживать;
  \item Высока стоимость установки;
  \item Высокая стоимость оборудования;
  \item Ограниченная информация.
\end{itemize*}
%

Программные методы классификации в основном полагаются на анализ видеопотока. В то время как такие методы лишены большинства недостатков представленных выше, они все еще имеют ограничение по быстродействию и точности обработки. В данной работе будут затронуты исключительно программные методы классификации.

В контексте данной работы стоит задача реализации системы детектирования классов транспортных средств. При решении данной задачи принимаются во внимание такие факторы системы как соответствие требованиям по количеству необходимых для обнаружения классов, устойчивости к изменяющейся на дороге ситуации, скорости обработки видеопотока и применимости к встраиваемым вычислительным системам.
	
Для соответствия требованиям по обнаруживаемым классам необходимо чтобы реализованная система была в состоянии классифицировать такие типы транспортных средств как:
%
\begin{itemize*}
  \item Легковой автомобиль;
  \item Грузовой автомобиль;
  \item Мотоцикл;
  \item Автобус;
  \item Трамвай;
  \item Троллейбус.  
\end{itemize*}
%

Помимо требований к способности распознавания указанных выше классов, необходимо также, чтобы реализованная система была устойчива к изменяющейся на дороге ситуации. Такие ситуации включают в себя смену времени суток, погодных условий, плотности движения транспортного потока и его направления. В случае если в подобных ситуациях уровень точности классификатора значительно снижается, то данная система не может быть использована для постоянного мониторинга сложившейся на дороге ситуации. 

Так как целью реализуемой системы является классификация транспортных средств в режиме реального времени, то возникает необходимость во введении ограничений по максимально возможному времени обработки кадра. Помимо того, необходимо принимать во внимание то, что данная система будет использоваться на встраиваемом устройстве, что также ставит ограничения на спектре возможных используемых технологий при решении поставленной задачи.
