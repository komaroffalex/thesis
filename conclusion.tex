%%%%%%%%%%%%%%%%%%%%%%%%%%%%%%%%%%%%%%%%%%%%%%%%%%%%%%%%%%%%%%%%%%%%%%%%%%%%%%%%
\conclusion
%%%%%%%%%%%%%%%%%%%%%%%%%%%%%%%%%%%%%%%%%%%%%%%%%%%%%%%%%%%%%%%%%%%%%%%%%%%%%%%%

В работе рассмотрена задача обнаружения и классификации транспортных средств. Приведены основные методы и алгоритмы для решения задач технического зрения по обнаружению и классификации объектов. Описаны достоинства и недостатки использования каждого из представленных методов для решения определенной задачи. 

По соотношению скорости обработки и устойчивости – обучаемые нейросетевые алгоритмы являются наиболее оптимальным вариантом для реализации детектора. Обнаружение объектов по моделям обладает существенными недостатками по сравнению с другими методами. Он требует составления большой базы моделей транспортных средств, и описания этих моделей с помощью геометрических примитивов. С учетом того, что некоторые типы транспортных средств могут быть очень похожи друг на друга – геометрические примитивы могут быть очень сложные для решения поставленной в этой работе задачи. Для реализации детектора на основе методов, представленных в разделе 1.2, необходимо изначально определить способ представления признаков объекта, а затем выполнить классификацию, с помощью метода опорных векторов. Обучаемые нейросетевые алгоритмы позволяют не определять признаки строго, так как основаны на сверточных нейронных сетях.

Использование классификаторов является одним из наиболее часто используемых способов решения задач, связанных с мониторингом транспорта. Применение таких алгоритмов не требует особых условий освещенности или стационарности камеры. Однако существенным минусом данного подхода является сложность подбора нужного классификатора и его обучение. Для обучения классификатора также необходима база обучающих примеров, что является существенным недостатком.

В работе также рассмотрены различные структуры моделей сетей глубокого обучения. Приведены основные способы расчета и оценки точности обнаружения объектов в зависимости от параметров кадра. 

Приведены блок-схемы алгоритмов дообучения и обнаружения, а также описан процесс дообучения модели, Реализованы программы дообучения модели на стационарном устройстве и запуска этой модели на встраиваемом устройстве в режиме рального времени. Также составлены наборы данных для дообучения модели и тестирования. Для решения поставленной задачи был выбран итеративный подход к дообучению сети. Если по результатам эксперимента не была достигнута желаемая точность, то дообучение производилось повторно с другим набором данных, или с другими параметрами. 

Для устойчивой классификации объектов типа «автобус» и «троллейбус» написан метод дополнительной классификации. Отдельно произведена классификация с помощью дескрипторов ключевых точек для всех меток «автобус» с целью определения класса принадлежности этого объекта. Протестированы различные методы обнаружения ключевых точек – ORB, BRISK, AKAZE.

