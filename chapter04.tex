%%%%%%%%%%%%%%%%%%%%%%%%%%%%%%%%%%%%%%%%%%%%%%%%%%%%%%%%%%%%%%%%%%%%%%%%%%%%%%%%
\chapter{Анализ полученных результатов}
%%%%%%%%%%%%%%%%%%%%%%%%%%%%%%%%%%%%%%%%%%%%%%%%%%%%%%%%%%%%%%%%%%%%%%%%%%%%%%%%

В данном разделе проводится анализ экспериментальных результатов, полученных в разделе 3. Эксперименты были проведены с различными конфигурациями системы, которые включают в себя:

%
\begin{itemize*}
  \item Система с исходной моделью распознавания объектов
  \item Система с моделью, дообученной на наборе данных, исключая метки «троллейбус»
  \item Система с моделью, дообученной на полном наборе данных	  
  \item Система с моделью, дообученной на наборе данных без меток «троллейбус», с дополнительной классификацией	
\end{itemize*}
%

По результатам проведенных экспериментов можно сделать вывод о том, что решение поставленной задачи обнаружения типов транспортных средств исключительно с помощью детектора на основе SSD Inception недостаточно эффективно. Для решения поставленной задачи использовалась дополнительная обработка области, помеченной как «автобус». Так как характерным признаком троллейбуса является использование электрических рельс, то становится возможным использование дополнительного отдельного дескриптора ключевых точек для обнаружения устройств энергопотребления троллейбуса. 

Анализируя метрики полноты, вычисленные для каждой конфигурации системы отдельно, может быть выявлено, что увеличение числа объектов на кадре приводит к снижению полноты обнаружения системы в целом. Это объясняется большим количеством транспортных средств, которые частично перекрывают друг друга. При таком расположении объектов становится трудно выделить их ключевые признаки, что приводит к снижению полноты обнаружения. 

Полнота обнаружения также имеет тенденцию снижаться для объектов, имеющих малую площадь на кадре. Это объясняется тем, что дальние от камеры транспортные средства чаще присутствуют на кадре частично закрытыми, что также не позволяет выделить их ключевые признаки верно. Стоит также отметить, что со снижением размера объектов на кадре заметно снижается не только полнота обнаружения, но и точность. 

Анализируя изменения метрик полноты обнаружения транспортных средств между различными конфигурациями системы, можно сделать вывод о том, что дообучение модели позволило достичь более высоких показателей полноты обнаружения в целом. Это объясняется наличием в обучающем наборе значительного количества изображений транспорта, расположенного на различном расстоянии от камеры, а также под разными углами обзора. Также стоит отметить, что значительная часть изображений транспортных средств получена в ночное время суток, что также позволило увеличить устойчивость системы к переменам освещения на кадре. 

Снижение средней точности обнаружения объектов при дообучении на полной выборке, включающей в себя объекты всех классов, перечисленных в функциональных требованиях к данной работе, объясняется плохим качеством классификации троллейбусов в целом. Так как при расчете метрик точности и полноты модели учитываются только метки объектов, на которых производилось дообучение, то точность и полнота обнаружения модели, обученной без троллейбусов значительно выше той, которая была обучена с учетом троллейбусов.

Анализируя результаты, полученные при использовании системы с дополнительной классификацией объектов можно наблюдать существенный прирост точности и полноты обнаружения в сравнении с моделью, обученной на полном обучающем наборе. Это объясняется более высоким количеством правильных классификаций меток «троллейбус» и «автобус». С другой стороны, использование в системе модели, обученной на полном обучающем наборе, приводит к снижению средней точности обнаружения в целом, даже при сравнении с исходной моделью. Это является результатом не только низкого количества правильных классификаций троллейбусов, но также и повышением количества неверных классификаций автобусов.

При сравнительном анализе графика средней точности обнаружения объектов в зависимости от тестовой подвыборки можно наблюдать повышение устойчивости модели в ситуациях, где наблюдается ночное движение транспорта, а также движение транспорта от камеры. Однако модель, дообученная на полной обучающей выборке не демонстрирует повышения точности, так как повышение устойчивости обнаружения транспорта ночью и сзади компенсируется одновременным понижением точности классификации автобусов и троллейбусов.

Согласно таблице, содержащей сведения о влиянии используемого фильтра и детектора ключевых точек на точность распознавания троллейбусов, наиболее оптимальной комбинацией является использование фильтра Лапласа и детектора ORB. Несмотря на то, что использование алгоритма Кэнни для выявления контуров объектов в связке с детектором ORB также показывает хорошую точность классификации, он является более затратным с точки зрения используемых ресурсов, так как наблюдается значительное снижение скорости обработки кадров при его использовании.










