%%%%%%%%%%%%%%%%%%%%%%%%%%%%%%%%%%%%%%%%%%%%%%%%%%%%%%%%%%%%%%%%%%%%%%%%%%%%%%%%
\chapter{Экспериментальные исследования разработанной системы на Jetson Nano}
%%%%%%%%%%%%%%%%%%%%%%%%%%%%%%%%%%%%%%%%%%%%%%%%%%%%%%%%%%%%%%%%%%%%%%%%%%%%%%%%

%%%%%%%%%%%%%%%%%%%%%%%%%%%%%%%%%%%%%%%%%%%%%%%%%%%%%%%%%%%%%%%%%%%%%%%%%%%%%%%%
\section{Планирование экспериментальных исследований}
%%%%%%%%%%%%%%%%%%%%%%%%%%%%%%%%%%%%%%%%%%%%%%%%%%%%%%%%%%%%%%%%%%%%%%%%%%%%%%%%

В качестве эксперимента в данной работе представлены результаты запуска системы обнаружения типов транспортных средств с использованием разных моделей обнаружения, дообученных в разделе 2.2.2. Помимо этого, при тестировании используется модуль дополнительной классификации, влияние которого на общую производительность системы будет измерено в этом разделе.

В качестве оценки точности предлагается использовать метрику точности, описанную в разделе 2.2.2. Точность рассчитывается как для каждого класса объектов отдельно, так и для всех классов суммарно. Структура тестовой подвыборки для эксперимента описана в разделе 1.3.5.

Для проведения эксперимента по оценке точности на тестовой подвыборке был отключен модуль получения видеопотока с USB камеры, и вместо этого, кадры берутся из директории с тестовыми изображениями. После этого, результаты обнаружения (метки классов и координаты) записываются в отдельный файл, после чего выполняется расчет метрик точности.

После того как дообучение модели было завершено, необходимо выполнить конвертацию в формат TensorRT для запуска модели на Jetson. Для этого используется файл этапа обучения, который обычно состоит из трех файлов:

\fbox{%
  \parbox{\textwidth}{
    model.ckpt-\$\{CHECKPOINT\_NUMBER\}.data-00000-of-00001\\
    model.ckpt-\$\{CHECKPOINT\_NUMBER\}.index\\
	model.ckpt-\$\{CHECKPOINT\_NUMBER\}.meta
  }%
}

После того как был выбран файл, содержащий подходящий этап, необходимо выполнить команду:

\fbox{%
  \parbox{\textwidth}{
    python object\_detection\/export\_inference\_graph.py\\
    --input\_type=\$\{INPUT\_TYPE\}\\
    --pipeline\_config\_path=\$\{PIPELINE\_CONFIG\_PATH\}\\
    --trained\_checkpoint\_prefix=\$\{TRAINED\_CKPT\_PREFIX\}\\
    --output\_directory=\$\{EXPORT\_DIR\}
  }%
}

Результатом выполнения программы будут являться файлы:

%
\begin{itemize*}
  \item Фиксированный граф модели
  \item Конфигурационный файл модели
  \item Исходный файл этапа обучения модели
\end{itemize*}
%

Экспортированный файл модели далее необходимо передать в качестве входного аргумента в программе обнаружения объектов на Jetson. После того как будет выполнена десериализация модели в формат TensorRT Engine – начнется выполнение обнаружения. 

%%%%%%%%%%%%%%%%%%%%%%%%%%%%%%%%%%%%%%%%%%%%%%%%%%%%%%%%%%%%%%%%%%%%%%%%%%%%%%%%
\section{Метрики точности обнаружения и классификации}
%%%%%%%%%%%%%%%%%%%%%%%%%%%%%%%%%%%%%%%%%%%%%%%%%%%%%%%%%%%%%%%%%%%%%%%%%%%%%%%%

Анализ точности модели производился с помощью набора метрик, описанного в разделе 2.1. В частности, использовались такие оценки как:

%
\begin{itemize*}
  \item Средняя точность (AP) в зависимости от пересечения:
	%
	\begin{itemize*}
	  \item При пересечении областей не менее 50\%
	  \item При пересечении областей не менее 75\%
	  \item При пересечении областей не менее 90\%	  
	\end{itemize*}
	%
  \item Средняя точность (AP) в зависимости от размера области:
  	%
	\begin{itemize*}
	  \item При размере областей менее \(32^2\)
	  \item При размере областей более \(32^2\) но менее \(96^2\)
	  \item При размере областей более \(96^2\)
	\end{itemize*}
	%
  \item Средняя полнота (AR):
  	%
	\begin{itemize*}
	  \item При максимальном количестве объектов равным 1
	  \item При максимальном количестве объектов равным 10
	  \item При максимальном количестве объектов равным 100
	\end{itemize*}
	%
  \item Средняя полнота (AR) в зависимости от размера области:
  	%
	\begin{itemize*}
	  \item При размере областей менее \(32^2\)
	  \item При размере областей более \(32^2\) но менее \(96^2\)
	  \item При размере областей менее \(96^2\)
	\end{itemize*}
	%
\end{itemize*}
%

Для визуализации точности модели в данном разделе используется кривая точности и полноты. Данная кривая является методом оценки успешных обнаружений и показывает зависимость точности и полноты от различных порогов классификации. Общая площадь под кривой пропорциональна и высокой точности и высокой полноте, в то время как высокая точность соответствует низкой частоте ложно-положительных классификаций, а высокая полнота соответствует низкой частоте ложно-отрицательный классификаций. В случае если оба показателя имеют высокие значения, это означает что классификатор корректно определяет тип объекта, и находит большинство этих объектов в выборке.

%%%%%%%%%%%%%%%%%%%%%%%%%%%%%%%%%%%%%%%%%%%%%%%%%%%%%%%%%%%%%%%%%%%%%%%%%%%%%%%%
\section{Экспериментальные исследования системы с исходной моделью}
%%%%%%%%%%%%%%%%%%%%%%%%%%%%%%%%%%%%%%%%%%%%%%%%%%%%%%%%%%%%%%%%%%%%%%%%%%%%%%%%

Для оценки результатов дообучения модели и дополнительной классификации реализованной системы в этом разделе проведено тестирование исходной модели, взятой из репозитория TensorFlow. Ниже приведены метрики точности и полноты обнаружения модели в зависимости от размера объектов и их количества на кадре:

\begin{table}[H]
	\def\arraystretch{1.3}
	\caption{Соответствие версий TensorFlow и TensorRT}
	\begin{center}
		\begin{tabular}{|c|c|}
			\hline
			Метрика & Значение\\  \hline			
			\(mAP^{50\%}\) & 0.53\\ \hline			
			\(mAP^{75\%}\) & 0.48\\ \hline
			\(mAP^{90\%}\) & 0.22\\ \hline
			\(mAP^{S<32}\) & 0.34\\ \hline
			\(mAP^{32<S<96}\) & 0.48\\ \hline
			\(mAP^{S>96}\) & 0.53\\ \hline
			\(mAP^{N=1}\) & 0.67\\ \hline
			\(mAP^{N<10}\) & 0.57\\ \hline
			\(mAP^{N<100}\) & 0.53\\ \hline
			\(mAP^{S<32}\) & 0.21\\ \hline
			\(mAP^{32<S<96}\) & 0.78\\ \hline
			\(mAP^{S>96}\) & 0.87\\ \hline			
		\end{tabular}
		\label{tabular:tab_examp}
	\end{center}
\end{table}

Для визуализации качества обнаружения объектов ниже представлена зависимость точности и полноты обнаружения от порога классификации:





























