
\keywords{%
  IT-ТЕХНОЛОГИИ, 
  МАШИННОЕ ОБУЧЕНИЕ, 
  ВСТРАИВАЕМЫЕ СИСТЕМЫ, 
  ПОКАДРОВАЯ ОБРАБОТКА, 
  ОБНАРУЖЕНИЕ ОБЪЕКТОВ, 
  ОБНАРУЖЕНИЕ ТРАНСПОРТНЫХ СРЕДСТВ, 
  ДЕТЕКТОР, 
  ОБРАБОТКА В РЕЖИМЕ РЕАЛЬНОГО ВРЕМЕНИ
}

\abstractcontent{
Объектом работы являются методы и алгоритмы обнаружения и классификации объектов на изображениях.
Цель работы - разработка системы обнаружения транспортных средств в режиме реального времени с использованием алгоритмов машинного обучения.
В процессе работы была реализована программа для обнаружения транспортных средств и проводились экспериментальные исследования её производительности.
В результате разработки была использована модель глубокого обучения по архитектуре SSD. Производилось дообучение модели на различных обучающих выборках. Использованные наборы изображений: собственный и ImageNet.
Измерены основные показатели точности системы для каждой обучающей выборки и при различных конфигурациях системы. 
Эффективность системы определяется точностью и скоростью её работы. Данное программное решение может применяться для реализации встраиваемых систем видеоаналитики транспортной ситуации на дорогах.
}

\keywordsen{
IT-TECHNOLOGIES, 
MACHINE LEARNING, 
EMBEDDED SYSTEMS, 
PER-FRAME PROCESSING, 
OBJECT DETECTION, 
VEHICLE DETECTION, 
DETECTOR, 
REAL-TIME PROCESSING
}

\abstractcontenten{
This work focuses on methods for detection and classification of objects on frames.
The goal of this work, is to develop the system for real-time vehicle detection using the machine learning methods.
In the course of this work was developed a program for detecting vehicles of different types. Also, the experiments were conducted to estimate its effectiveness.
As a result, the SSD deep learning model was used. The transfer learning was made using different training subsets. Subsets contain original images as well as images from ImageNet dataset.
The main indicators of system accuracy were measured for each training subset and for various system configurations.
The effectiveness of the system is determined by the accuracy and speed of its operation. This software solution can be used to implement embedded video analytics systems to monitor road traffic situation.
}
